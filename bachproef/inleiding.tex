%%=============================================================================
%% Inleiding
%%=============================================================================

\chapter{\IfLanguageName{dutch}{Inleiding}{Introduction}}%
\label{ch:inleiding}

Door de evolutie van computers en dus ook browsers in het algemeen maken web developers steeds meer gebruik van 3D elementen in hun projecten. Deze elementen stellen de gebruiker in staat om op een compleet nieuwe manier om te gaan met web applicaties. 

\section{\IfLanguageName{dutch}{Probleemstelling}{Problem Statement}}%
\label{sec:probleemstelling}

De implementatie van 3D elementen in web applicaties zal ongetwijfeld een toename kennen in de jaren die volgen. Het framework Three.js is een van de koplopers in dit domein en bestaat al een eind maar is pas de voorbije jaren, mede door de algemene stijging in grafische kracht van computers, in populariteit toegenomen. 

Gebruikers kunnen a.d.h.v. Three.js dus met meer interactie omgaan op het web. Dit kan goede maar ook slechte gevolgen hebben, dit is waar het onderzoek een zo goed mogelijk antwoord zal proberen te formuleren. De focus van het onderzoek zal bij e-commerce liggen, hiermee kunnen developers een beter beeld krijgen of de gevolgen van het implementeren van Three.js een meerwaarde kan bieden voor hun web applicaties.

\section{\IfLanguageName{dutch}{Onderzoeksvraag}{Research question}}%
\label{sec:onderzoeksvraag}

In dit onderzoek wordt een antwoord gezocht op de volgende vragen:

\begin{itemize}
	\item[1.] Wat zijn de voordelen en beperkingen van het gebruik van 3D-interactie in e-commerce, vergeleken met traditionele 2D-interfaces?
	
\end{itemize}

\section{\IfLanguageName{dutch}{Opzet van deze bachelorproef}{Structure of this bachelor thesis}}%
\label{sec:opzet-bachelorproef}

De rest van deze bachelorproef is als volgt opgebouwd:

In Hoofdstuk~\ref{ch:stand-van-zaken} wordt een overzicht gegeven van de stand van zaken binnen het onderzoeksdomein, op basis van een literatuurstudie.

In Hoofdstuk~\ref{ch:methodologie} wordt de methodologie toegelicht en worden de gebruikte onderzoekstechnieken besproken om een antwoord te kunnen formuleren op de onderzoeksvragen.


% TODO: Vul hier aan voor je eigen hoofstukken, één of twee zinnen per hoofdstuk

In Hoofdstuk~\ref{ch:proofofconceptConventioneel}, 

In Hoofdstuk~\ref{ch:proofofconceptThreeJS}, 

In Hoofdstuk~\ref{ch:conclusie}, tenslotte, wordt de conclusie gegeven en een antwoord geformuleerd op de onderzoeksvragen. Daarbij wordt ook een aanzet gegeven voor toekomstig onderzoek binnen dit domein.