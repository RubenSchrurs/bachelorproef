%%=============================================================================
%% Conclusie
%%=============================================================================

\chapter{Conclusie}%
\label{ch:conclusie}

% TODO: Trek een duidelijke conclusie, in de vorm van een antwoord op de
% onderzoeksvra(a)g(en). Wat was jouw bijdrage aan het onderzoeksdomein en
% hoe biedt dit meerwaarde aan het vakgebied/doelgroep? 
% Reflecteer kritisch over het resultaat. In Engelse teksten wordt deze sectie
% ``Discussion'' genoemd. Had je deze uitkomst verwacht? Zijn er zaken die nog
% niet duidelijk zijn?
% Heeft het onderzoek geleid tot nieuwe vragen die uitnodigen tot verder 
%onderzoek?


Bij de start van de tweede POC was er veel ambitie voor een volledig interactieve website te maken. Bij nader inzien bleek dit dus niet zo gemakkelijk te zijn. Alhoewel de theorie van Three.js niet uiterst moeilijk is, is de toepassing ervan echter zeer complex. Het verg

\begin{*omgeving-naam*}
	\item[2.] Hoe beïnvloedt 3D-interactie de betrokkenheid en tevredenheid van gebruikers in een e-commerce context?
\end{*omgeving-naam*}