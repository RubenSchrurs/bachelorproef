%---------- Inleiding ---------------------------------------------------------

\section{Introductie}%
\label{sec:introductie}

Een van de recente trends als gevolg van de stijging in e-commerce is het gebruik van 3D Javascript frameworks. Deze frameworks bieden web-developers de mogelijkheid aan om 3D interactie te implementeren in hun projecten. Vrijwel een van de meest gebruikte frameworks is Three.js die ons toelaat om 3D objecten te manipuleren in onze browser. Hoewel er een breed aanbod is aan dit soort frameworks gaan web-developers en daaropvolgend bedrijven, niet snel grijpen naar een website die gebruik maakt van 3D interactie.

In deze bachelorproef wil ik dus nagaan of het gebruik van 3D interactie in webshops een positieve impact heeft op het gedrag van gebruikers. Om de invloeden en eventuele voor en/of nadelen in kaart te brengen zal er gebruik gemaakt worden van de UX metrics. Voor de proof-of-concept zullen twee webshops met dezelfde inhoud gemaakt worden, eenmaal op een conventionele manier en eenmaal in 3D gebruikmakend van Three.js. De UX metrics zullen voor beide webshops worden toegepast om zo tot een conclusie te komen welke vorm van interactie bij een webshop superieur is. Met het gevonden resultaat kunnen developers gerichter hun keuze maken over welk soort webshop ze willen maken.

%---------- Literatuurstudie ---------------------------------------------------

\section{Literatuurstudie}%
\label{sec:literatuurstudie}

Over de laatste jaren zijn apparaten en browsers een stuk sterker geworden, dit laat ons toe om 3D renders te maken. Een van de manieren om dit te doen is door WebGL te gebruiken. WebGL is een technologie die ons toelaat om te tekenen, weer te geven en te communiceren met gesofisticeerde interactieve 3D graphics \autocite{Matsuda2013}. Gebruik maken van WebGL is echter zeer complex doordat een diepe kennis van de innerlijke details van WebGL  nodig is, alsook het beheersen van een complexe shader-taal \autocite{Dirksen:2015aa}. Hierdoor is WebGL dus voor veel web-developers, en vaak ook designers, niet gemakkelijk te leren. Dit is waar Three.js in het verhaal terecht komt, om web-developers de mogelijkheid te geven voor efficiënter en gemakkelijker overweg te kunnen met de WebGL library. Om die reden heeft Ricardo Cabello het framework Three.js gemaakt. Three.js beschikt over meerdere tekenmethodes en kan terugvallen op een 2D visualisatie indien WebGL niet ondersteund is door de browser \autocite{Danchilla2012}.

Three.js is dus een zeer krachtig framework om 3D scenes te creëren in een browser. Alhoewel het gemakkelijker en gebruiksvriendelijker is dan WebGL, is Three.js nog steeds niet de eerste keuze van web-developers om bijvoorbeeld een webshop te maken. Om een goed beeld te scheppen over welke soort methode de voorkeur zal krijgen bij gebruikers zal er gebruik gemaakt worden van UX metrics.

Statistieken komen overal voor in ons leven en dus ook bij de ervaring van een website-gebruiker. Alle UX metrics moeten kwantificeerbaar en waarneembaar zijn, zo kunnen UX metrics een hulpmiddel zijn om tot een geïnformeerde beslissing te komen en nieuwe inzichten te geven \autocite{Albert2013}.

De UX metrics die voor dit onderzoek zullen gebruikt worden kunnen we onderverdelen in twee categoriëen, namelijk performance metrics en special topics. Volgens \textcite{Albert2013} bestaan performance metrics uit: \begin{itemize}
	\item Task succes
	\item Time on task
	\item Errors
	\item Efficiency
	\item Learnability
\end{itemize}
En special topics metrics uit:
\begin{itemize}
	\item Live website data
	\item Card-sorting data
	\item Accesibility data
	\item Return-on-investment data
\end{itemize}

In deze bachelorproef zal een PoC opgeleverd worden, gebruikmakend van React voor front-end en Node.js voor back-end. React is een Javascript user library die ontwikkeld is door Facebook en het wordt gebruikt voor het maken van user interfaces \autocite{Banks2017}. Node.js kan worden omschreven als een runtime omgeving voor Javascript gebouwd bovenop Google's V8 engine \autocite{Satheesh2015}.
%---------- Methodologie ------------------------------------------------------
\section{Methodologie}%
\label{sec:methodologie}

In de eerste fase van de bachelorproef zal onderzoek gedaan worden naar Three.js a.d.h.v. een online cursus genaamd "Three.js Journey", die een zeventigtal uren van lessen omvat. In het tweede deel van de bachelorproef zal een lege webshop voorzien worden, daarna wordt de webshop eenmaal gevuld met producten die op een conventionele manier geïntegreerd zijn en eenmaal met 3D objecten gebruikmakend van Three.js. Voor de webshops zal gewerkt worden met een volledig identieke maar onafhankelijke back-end waarbij de inhoud van de front-end exact hetzelfde is om een zo correct mogelijk vergelijkende studie te realiseren.
 Na het uitvoeren van deze twee praktische onderzoeken zal een vergelijkende studie volgen tussen de twee webshops waarbij de UX metrics gebruikt worden als parameters. Hiervoor zal een steekproef door 10 of meer gebruikers afgenomen worden. De interacties van de participanten zullen worden geanalyseerd a.d.h.v. de UX metrics.

%---------- Verwachte resultaten ----------------------------------------------
\section{Verwacht resultaat, conclusie}%
\label{sec:verwachte_resultaten}

Uit dit onderzoek moet duidelijk blijken of het implementeren van 3D interactie met behulp van Three.js een positieve invloed heeft op de user experience in e-commerce. 

